\documentclass{article}

\usepackage[final]{nips_2017}

\usepackage[utf8]{inputenc} % allow utf-8 input
\usepackage[T1]{fontenc}    % use 8-bit T1 fonts
\usepackage{hyperref}       % hyperlinks
\usepackage{url}            % simple URL typesetting
\usepackage{booktabs}       % professional-quality tables
\usepackage{amsfonts}       % blackboard math symbols
\usepackage{nicefrac}       % compact symbols for 1/2, etc.
\usepackage{microtype}      % microtypography
\usepackage{authblk}
\usepackage{graphicx}
\usepackage{float}


\title{The Effects of Different Sampling Techniques on the SoundCloud Network}

\author{
	Josiah Buxton \hspace{45pt} Christopher Godley\\
	Department of Computer Science\\
	University of Colorado Boulder\\
	\texttt{josiah.buxton@colorado.edu} \\
	\texttt{christopher.godley@colorado.edu} \\
}

\begin{document}
	
\graphicspath{ {c:/users/jbuxofplenty/Github/soundcloud_crawler/final_report/} }

\maketitle

\begin{abstract}
  Data is becoming more freely available to any individual who is interested in it.  Wrangling this data into a meaningful, clean dataset where one can come to accurate conclusions on a topic is no simple task.  This paper is designed to explain different methods of sampling data on SoundCloud.  It will also illustrate the effects of these sampling techniques on the structure of a network generated from the data.  We programmed a web scraper that browses through the HTML pages under the SoundCloud umbrella and parses relevant data of users and their attributes into memory.  We then use this data to generate a network where SoundCloud users are the nodes and following and being followed by other users are considered edges.  We then perform different algorithms and calculate metrics on the generated network.  We've found that small changes in the data wrangling process can lead to drastic differences in the structure of the generated network.
\end{abstract}

\section{Background and Motivation}
\subsection{Definition}
There has been an explosion recently with research in the area of complex networks, particularly social networks.  Social networks are defined by Danah Boyd and Nicole Ellison as "web-based services that allow individuals to (1) construct a public or semi-public profile within a bounded system, (2) articulate a list of other users with whom they share a connection, and (3) view and traverse their list of connections and those made by others within the system [1]."  These networks can be represented with a graph structure where the nodes in the graph are characterized by users in the social network, and the edges are characterized by the connections in between the users.  These connections can vary from website to website, but normally they are symbolized by users being "friends" or "followers" of other users.  Being "friends" usually provides users with functions that allow for bidirectional communication with other users.  Being a "follower" usually entails unidirectional information exchange where the user being followed posts some sort of information which can be consumed by all of his/her "followers."  There are many types of popular social networks in existence today, for example SoundCloud, Facebook, Spotify, Twitter, and LinkedIn.  Each of these networks are specialized for a particular audience and provide the framework for exchanging information between users.  In this paper, we will be focusing on the social network SoundCloud, as it is a social network that caters to musicians and artists.

\subsection{Motivation}
Music has always been a major component of our lives and much time has been spent trying to connect with other people who share our passion.  It was only natural for us to try and look for easier, alternative methods to make positive connections with new music and musicians.  This intention provided the motivation for our project.  We wanted to use real user data from around the world in order to connect musicians to other musicians by providing them with recommendations of "like" users.  This link prediction could be used to connect musicians with potential fans or perhaps musicians with other like-minded musicians in order for them to collaborate on future projects.  It could also be used to connect fans with new music/musicians that they would enjoy or connect fans with other fans to attend different shows and make new connections.  Another potential use of grabbing the user data and analyzing it would give us insight as to which artist could potentially break out in the future and become a big name.

In order to accomplish this goal we planned on utilizing skills learned from our Network Analysis and Modeling class and our knowledge of social networks to represent the data that we pulled into a network structure.  The data we planned on grabbing was perfect for being represented in a network because nodes could be characterized as users on SoundCloud and edges could be represented by users "following" or being "followed" by other users.  There was no freely available API for us to use in order to gather this data so we opted to grab and clean the data ourselves by programming a customizable web-scraper to fulfill our needs.  As we began gathering data, we soon realized that the task of populating a clean sub dataset that accurately symbolizes the structure of the full network was not as easy as we thought.  Because of this, we chose to diverge from our initial goal of performing link prediction in our generated network to performing a variety of sampling techniques in order to illuminate the biases that are introduced into generated networks when sampled in different ways.

We now will begin with a description of our methods used to sample the data from the SoundCloud network and explanations of our thought processes that lead us to our design decisions.

\section{Experimental Setup}
\subsection{Programming Environment}
We mainly chose SoundCloud as our server to pull our data from because of the accessibility of the data.  We looked into a few other similar frameworks such as Apple Music, Spotify, and Google Play.  All of the frameworks offered API's but the none of the packages provided accessibility for the data we were after.  Mostly, the API's were used to implement an interface in which you could pull music from their server and interact with it in a variety of ways.  Because we were focused on user data, we then looked into the feasibility of pulling the data off of the user interfaces for each of the servers.  Both Apple Music and Spotify mainly use offline applications in order for users to interact with their server.  Google Play and SoundCloud on the otherhand, use a web interface for users to interact.  We wanted to utilize the web interface as there are many tools we were familiar with in order to grab this data easily.  We made our decision to work with SoundCloud because we both have used it in the past and it is known to house a diverse community of experienced and inexperienced artists alike. We believed the pool of inexperienced artists would give us more opportunities to predict when artists would "blow up." 

The next decision that we needed to make was which method to use to obtain the data itself.  We knew we would be programming in python because of the experience we gathered in the Fall semester of 2017 for our Network Analysis and Modeling class.  We researched into web scraping packages for python and came upon a few that looked very enticing to use (Scrapy, Beautiful Soup 4, Requests, Selenium, and lxml).  We began by choosing the two most promising ones for our particular application, Scrapy and Beautiful Soup 4.  After a few hours of struggling to learn the libraries, we opted to use UrlLib3 as it had a very intuitive interface.  The issue that arose from this decision was that UrlLib3 is only an HTTP client for python which provides a simple interface for grabbing HTML pages but it is not an all encompassing web scraper.  We were also both familiar with Regular Expression so we believed that we could easily parse the information needed after capturing the HTML page with UrlLib3.

\subsection{Sampling}

Upon making the decision of how to capture our data, we then needed to decide what sampling technique to use in order to acquire a subset of the network that would have a similar structure to the full size network.  We decided to use a type of seed-based sampling, called snowball sampling, using the "Top 50" song chart webpage as our seed page.  Snowball sampling is defined as "for each seed vertex \textit{i}, and distance \textit{l}, include all vertices (and their neighbors) for an \textit{l}-step breadth-first search tree rooted at \textit{i} [2]."  Each vertex \textit{i}, pertains to a particular artist which is obtained from grabbing the hyperlink to the artist's profile page on the seed page, navigating to the page, and then scraping the data.  Each step of \textit{l}, is defined by the navigation from the profile page of the artist to the profile pages of all the artists "connected" to the first artist.  A connection is whether the artist follows other users or a user follows the artist.  These connections can be obtained on the "followers" and "following" pages of the artist, which are other links on the profile page.  We then repeated the step of scraping an artist, navigating to his/her followers' and following' profile pages and scraping them \textit{n} times to populate the dataset for our graph.  Below is a visual representation of how we sampled the data and visualizations for the terminology used above.

\begin{figure}[h]
	\centering
	\includegraphics[scale=0.45]{sampling_visualization}
	\caption{Visualization of our first sampling technique.}
\end{figure}

We then had to decide what type of data to grab off of each of the SoundCloud users' profile pages.  There were many fields on the profile pages but a lot of the fields were optional when creating the page so some users did not have data for a particular field.  We ended up choosing five different fields labeled in Table 1.

\begin{table}[h!]
	\centering
	\begin{tabular}{||c c c||} 
		\hline
		Field & Datatype & Optional \\ [0.5ex] 
		\hline\hline
		id & int & False \\
		\hline
		href & string & False \\ 
		\hline
		num\_tracks & int & False \\
		\hline
		num\_followers & int & False \\
		\hline
		num\_following & int & False \\
		\hline
		city & string & True \\
		\hline
		latitude & float & True \\
		\hline
		longitude & float & True \\
		\hline
		followers & list & False \\
		\hline
		following & list & False \\
		\hline
	\end{tabular}
	\caption{Depiction of data scraped from each artist}
	\label{Table:1}
\end{table}

The first field was the "id" field, which was a unique integer assigned to each of the users in the dataset.  This was used to easily identify different users when we loaded them into the network.  The "href" field was a string that could be appended to a base soundcloud url in order to correctly request the HTML page from the server.  The "num\_tracks" field was the number of tracks the user had uploaded to his/her profile page.  This was either '> 0' if the user was an artist and '= 0' otherwise.  The "num\_followers" field was the number of followers a user had.  Similarly, the "num\_following" field was the number users the user was following.  The "city" field was a field that a user could enter a string into that could either represent their current place of residence or their hometown.  From the "city" field, we calculated the latitude and longitude fields.  The last two fields, "followers" and "following," were each lists containing the names of each of the other users the user was connected to.

Note that we initially chose these fields with the intent to perform link prediction on nodes in the network and predict whether an artist was about to rise quickly in popularity.  Based on issues explained in more detail in the results section, we used only a fraction of these fields in our analysis.

\subsection{Network}

After obtaining all of the data from SoundCloud and storing it into a dictionary, we needed to decide how to load it into a structure where we could perform our analysis on it. We opted to use the package NetworkX as we were very familiar with it and all of its functionality.  We made the significant design decision to keep the network's edges undirected versus directed because otherwise it would limit us in being able to use certain functions in the NetworkX package as well as making it more difficult to generate a densley connected graph given our time constraints.  We knew that this network seemed to generalize better to a directed network and therefore we coded in the option to generate one in the future.  

\section{Analysis}

After gathering the data to populate our network and loading it into NetworkX in order to conceptualize it, we realized very quickly that we were going to have to refine our method for sampling the data.  It was readily apparent that the graph generated was not very correlated to the structure of the full network.  Because most of the project was spent refining the sampling techniques, we decided to shift our focus from link prediction and rising artist analysis to analyzing the effect of different sampling techniques on the structure of the network generated.  In order to do so, we chose on using a consistent set of algorithms to test on the graph and consistent set of metrics to characterize the graph.  For algorithms, we chose the Guilt by Association heuristic and Link Prediction based on degree product, common neighbor, and shortest path heuristics.  For the metrics, we chose to use clustering coefficients, the total number of triangles, and degrees.  Each of these algorithms and metrics are discussed in more detail in their respected sections.

\subsection{Network Topologies}
The first network topology that we made was generated from our initial sampling design decision to
snowball outwards from the "Top 50" songs chart.  We initially grabbed each of the artists on this
page and then grabbed a subset of their followers and the users the artists were following.  We only
grabbed a subset (initally unbeknown to us) because we ran into a large roadblock where SoundCloud
had implemented an "infinite scrolling" feature on the "following" and "followers" pages attached to
a profile page.  This feature loaded only twenty different artists initially when you made a request
for the page until the user scrolled down to the bottom of the page, in which case twenty more
artists would be loaded.  This process would be repeated until all of the followers or following
were displayed.  After realizing this, we needed to implement a solution where we utilized a package
called Selenium. This package implements a chrome driver to allow programmatic control over the
chrome web application.  We then used javascript package called JQuery in order to send a request to
the SoundCloud server to send us the next twenty users in the followers and following web pages.  We
repeated this JQuery request a total of 10 times, stalling the script for a tenth of second in
between to give time for the server to process the request and to account for latency.  This still
only gave us a maximum of around 200 followers and 200 following.  We would've preferred to keep
this script running until all of the users were loaded but we didn't due to time constraints and the
sheer number of followers available. Since we were seeding with the largest artists on SoundCloud,
there could easily have been over million followers for each of them. This was just too large of a
network for our hardware to both scrape and process in the time alotted for the project.  Below in Figure 2 is a visual representation of our first network generated. 

\begin{figure}[h]
	\centering
	\includegraphics[scale=0.29]{take_1_visualization}
	\caption{Visualization of our first network topology.  Note that this network is scaled down to the first 500 nodes that were originally added.  The network consisted of around 21,000 nodes and 30,000 edges when we finished scraping and began to perform our algorithms and metrics on it.  The colors and numbers on the nodes are representing the number of tracks each of the users had.}
\end{figure}

The visualization of the network is exactly what we expected to see because of the sampling scheme that we used.  Each of the nodes in the visualization are colored based on the number of tracks attribute.  Thus, the cyan nodes are users that had the number of tracks attribute equal to zero.  You can see that there are many cyan nodes that fan out connections to a different colored node in the center.  These different colored nodes indicate the first 40-50 artists that we grabbed from the "Top 50" song charts seed webpage.  The cyan users indicate the followers and following of the initial artists.  There are also some cyan nodes in the middle that have multiple connections to a variety of different colored nodes.  We expected to see this because most of the "Top 50" songs were in the rap genre and therefore, we believed it would be more probable for the different artists in the "Top 50" to share different followers and following.  We did not believe that this small, subset-network was structurally similar to the full SoundCloud network.  This belief was the motivation to creating our second network topology.

The second network generated was the result of a change in our sampling technique.  We wanted the
network to have more connections between nodes because we believed it would be more similar to the
full network.  In order to do this, we thought we could adjust our sampling technique to adaptive
sampling.  We used our initial pickled users file after our script captured the nodes and edges from
"depth=1" (refer to Figure 1 for a better understanding).  We then only branched out from those users to the users they were following.  We believed this would give us a network that would contain many more artists rather than users because most artists have a larger following/followers ratio.  We were hoping this would lead us to a more connected network as well because we thought the initial followers of the "Top 50" artists would make connections to many more common artists since most of the "Top 50" artists were in the same genre.  Below is a depiction of the network generated from our second sampling technique.

\begin{figure}[h]
	\centering
	\includegraphics[scale=0.29]{take_2_visualization}
	\caption{Visualization of our second network topology.  Note that this network is scaled down using a function that removed any of the users that did not upload any tracks to the profile page (387 nodes).  The network consisted of around 40,000 nodes and 60,000 edges when we finished scraping and began to perform our algorithms and metrics on it.  The colors and numbers on the nodes are representing the number of tracks each of the users had.}
\end{figure}

While this visualization shows a very connected network, it is hard to compare this graph to the first graph.  We did not create a visualization in exactly the same manner as the first visualization because it would've given us the exact same figure.  This is because we used the pickled file of initial users from the first sampling in order to provide the seed for our second graph.  If we used this and did the same process of graphing the first 500 nodes added, they would've been identical figures.  In order to get a different visualization of the network, we made a function that "cleaned" the dataset and removed any of the users that did not upload any tracks.  We believed that this dataset would then only entail the "artists" in our dataset.  Even so, after analyzing our network we still believed the network was not very structurally similar to the full SoundCloud network so this motivated us to adjust our sampling technique even further.

For our third network, we wanted to encompass a more representative sample of all of the artists on the network and not just the artists from the rap genre.  In order to do this, we utilized the adaptive sampling technique from our second take to only grab the followers of the users from the initial seed.  We also changed our initial seed of users to contain the artists of the "Top 50" songs from each of the genres labeled on SoundCloud.  There were 30 different genres total, represented in the table below.  Each of these HTML pages were very similar to the first "Top 50" songs page where we gathered our initial seed for the first two sampling techniques, so it was quite easy to adjust our scripts in order to populate our new seed of users.  Below is a visualization of our third network generated from this new sampling technique.

\begin{figure}[h]
	\centering
	\includegraphics[scale=0.43]{take_3_visualization}
	\caption{Visualization of our third network topology.  Note that this network is scaled down to the first 500 nodes that were originally added.  The network consisted of around 6,000 nodes and 6,000 edges when we finished scraping and began to perform our algorithms and metrics on it.  The colors and numbers on the nodes are representing the number of tracks each of the users had.}
\end{figure}

In this network, we were surprised to see the number of components in the graph greatly increase from the previous network.  We believed that changing the sampling technique to only gather the users that were followed by a particular user would keep the network very connected.  Upon further investigation, we now understand that the number of components is due to the different genres of music an artist is apart of.  It is reasonable to believe that there are different communities of users within that can be discerned from different genres as people tend to enjoy one or a few different genres rather than all of the genres in a given spectrum.  It is also apparent that most of the nodes in our graph represent users and not artists as the color of the majority of nodes is cyan blue (num\_tracks=0).  We thought we were on the right track to getting a graph structurally similar to the full network so we decided to use our graph generated using this sampling technique and build upon it further.

Our final topology that we created was just an extension of the third topology in the sense that we took the users that were present in the network and added attributes on the nodes themselves and edges between them.  We also expanded the user base using the same sampling technique in our third topology.  We spent the most time generating this final network and also added more iterations to the scrolling in the "following" and "followers" pages, to get a network more structurally simlilar to the full network.  Below is a visualization of our final topology.

\begin{figure}[h]
	\centering
	\includegraphics[scale=0.29]{take_4_visualization}
	\caption{Visualization of our fourth network topology.  Note that this network is scaled down using a function that removed any of the users that did not upload any tracks to the profile page and then randomly sampled further down to 350 nodes so as to be comparable to the 2nd topology configuration.  The network consisted of around 30,000 nodes and 50,000 edges when we finished scraping and began to perform our algorithms and metrics on it.  The colors and numbers on the nodes are representing the number of tracks each of the users had.}
\end{figure}

\subsection{Algorithms}

\subsubsection{Graph Metrics}
The table below contains all of the basic graph metrics that we gathered from each of our first three topologies. The number of nodes, number of edges, number of triangles, max clustering coefficient, overall clustering coefficient, max degree, and mean degree were all measured. The nodes and edges are provided to give insight to the size of the graph, while the triangles and clustering coefficients speak to the community structure of the network. We can see that each graph had a rather low clustering coefficient despite each having nodes with a max value of one. The third topology was the worst performing here, as we could have predicted given the shape of that topology. The second was surprisingly lower than the first, however, as we expected the second topology to be much more connected than the first. It did have a higher mean degree than the first, but all were still fairly low values. 

\begin{table}[h!]
	\centering
	\begin{tabular}{||c c||} 
		\hline
		& Take 1 & Take 2 & Take 3\\ [0.5ex] 
		\hline\hline
		N & 21258 & 39734 & 5598 \\
		\hline
		Number of Triangles & 11715 & 11229 & 411 \\ 
		\hline
		Max Clustering Coeff & 1 & 1 & 1 \\
		\hline
		Clustering Coeff & 0.023 & 0.012 & 0.0095 \\
		\hline
		Max Degree & 183 & 184 & 157 \\
		\hline
		Mean Degree & 2.75 & 3.02 & 2.15 \\
		\hline
	\end{tabular}
	\caption{Graph metrics taken from the first 3 topologies. }
	\label{Table:2}
\end{table}

\subsubsection{GbA Heuristic}
The GbA (Guilt by Association) heuristic is an approach to gauge the associativity of a graph given a certain attribute.  That is to say, how likely nodes are to connect to other nodes with the same attribute.  We wanted to use this heuristic to see how associative artists were to non-artists on the SoundCloud domain. In order to do this, we used code from one of our previous homeworks in Network Analysis where we had to gauge whether the \textit{var} gene sequences (the nodes in the network) in the human malaria parasite exhibited homophily with respect to the cys/PolV attribute attached to the nodes.  We wanted to use this code in order to gauge whether the graphs we generated were assortative based on the number of tracks the users uploaded to their profile pages.  More specifically, we wanted to see if the graph was assortative based on artist/non-artists. Below is a graph that demonstrates the code applied to our fourth topology.

\begin{figure}[h]
	\centering
	\includegraphics[scale=0.29]{take_4_gba}
	\caption{Graph that demonstrates the GbA heuristic applied to the num\_tracks attribute of nodes vs the fraction of nodes observed with the correct attribute.  This depiction is of the heuristic performed on the fourth topology and each point was averaged over 10 different iterations.  Note that applying this heuristic to all of our topologies produced very similar results.}
\end{figure}

We were having a very difficult time trying to understand why the graph turned out the way it did.  When analyzing the GbA heuristic algorithm, we realized that we applied it wrongly to our application.  In order for the algorithm to predict a label, it first looks at all of the other labels of its neighbors.  It then finds the majority label and predicts it to be the actual label.  The reason this heuristic was wrongly applied was because the number of tracks uploaded by a user is very specific to the user and it is not logical to think that if your neighbors have produced a certain number of tracks, you should produce the same amount of tracks.  Therefore, we decided to implement a variation of the algorithm where we lumped the num\_tracks attribute into a new binary attribute where 1 represented that you've uploaded 1 or more tracks and 0 otherwise.  Below is a depiction of this algorithm applied to our fourth topology.

\begin{figure}[h]
	\centering
	\includegraphics[scale=0.29]{take_4_gba_2}
	\caption{Graph that demonstrates the GbA heuristic applied to the new binary num\_tracks attribute of nodes vs the fraction of nodes observed with the correct attribute.  This depiction is of the heuristic performed on the fourth topology and each point was averaged over 10 different iterations.  Note that applying this heuristic to all of our topologies produced very similar results.}
\end{figure}

The results depicted above leads us to believe that the graphs that we generated are very homopholic with respect to artists/non-artists.  Intuitively, we expected the opposite to be true when applying this heuristic to the full network.  This is because it is our belief that the majority of connections would be from non-artists following artists.  The results could either represent a heavy bias that is introduced into the network based on our sampling techniques or a fundamental error in our algorithm that we performed.  In order to test this hypothesis, we wrote a function that checked to see what types of edges existed in our graph.  The results of the function are depicted in the table below.

\begin{table}[h!]
	\centering
	\begin{tabular}{||c c||} 
		\hline
		Type of Edge & Number of Edges \\ [0.5ex] 
		\hline\hline
		Users to Users & 9161 \\
		\hline
		Users to Artists & 36302 \\ 
		\hline
		Artists to Artists & 3190 \\
		\hline
		Total Edges & 48653 \\
		\hline
	\end{tabular}
	\caption{Types of edges in the fourth topology}
	\label{Table:3}
\end{table}

The table above gives a clear indication that there is a serious error in the implementation of the GbA Heuristic on the SoundCloud network.  This could've also been the case as to why we generated mostly identical graphs for each one of the topologies.  Unfortunately, we did not have enough time before the submission of this paper to rectify the issue.

\subsubsection{Link Prediction}
Link prediction is the task of resolving missing edges or adding new edges based on the structure of the networks nodes or attributes. There are many different methods and applications for link prediction, but here we focus on applying and comparing three algorithms: (1) Degree Product AUC, (2) Common Neighbors AUC, and (3) Shortest Path AUC. Each of these algorithms can be represented by a score if you let \Gamma(i) denote the set of neighbors of i in the network, and let \sigma(i, j) be the length of a geodesic path between i and j. The scores for the Degree Product AUC are represented by the product of the two nodes' degrees, just as the name would suggest. The Common Neighbors AUC score is found by taking the intersection of the neighbors of nodes i and j, divided by the union of the neighbors of nodes i and j. To find a Shortest Path AUC score, we simply calculate 1/\sigma(i, j). 

Each of the three AUC metrics were performed on the four topologies we generated. Despite the significantly different design decisions that built each network, the plots generated from each of these are surprisingly similar. Below are the plots generated from the second and fourth topologies, as these were our most promising networks that were generated. Comparing those two figures, the fact that the Shortest Path AUC was the most successful metric indicates that these topologies had more in common than we originally thought. It could also indicate that this may be the correlating performance to the full SoundCloud network. The common neighbors metric was always the worst performing, which is also interesting. Looking back to how this metric is scored, one might expect that the lack of performance here ties into a lack of correlation between what nodes are nearby and that neighbors are not good indicators of links. Intuitively this is a little surprising, as one might have expected musicians to cluster with friends and connect among like friends. This result seems to indicate the opposite and that a network of musicians is more complex. To try and explain why the Shortest Paths AUC is most accurate, perhaps it would be correct to assume that artists connect with other artists when convenient. Another explanation might be that artists are trying to network and build in the most efficient way possible, which means optimizing who they connect with based on path length rather than community.  

\begin{figure}[h]
	\centering
	\includegraphics[scale=0.29]{take_2_lr}
	\caption{Representation of link prediction over the second network topology. The Shortest Path AUC overtakes the other two to be the best performing after 50\% of the edges are observed.}
\end{figure}

\begin{figure}[h]
	\centering
	\includegraphics[scale=0.29]{take_4_lr}
	\caption{Representation of link prediction over the fourth network topology. Overall very similar to the graph generated by the second topology. The Shortest Path AUC is still the best, but this time it is from approximately after 30\% of edges are observed, rather than 50\%.}
\end{figure}

\section{Future Work}
This was an insightful project on sampling and constructing a large dataset that is a representative subset of a complete network. By the end, we felt that we were just beginning to see promising results and that our sampling methods were maturing. There is a lot of work left to do on this project regarding the link prediction and artist trending prediction. We could easily spend dozens of more hours generating better networks alone, but we feel that with the lessons we've learned that we could sample SoundCloud just once more and have a reliable network to base our assumptions on. In a final revision of our topology, it would be best if we incorporated a larger sample of non-artist users. These users provide the glue of interconnectivity to our model and would allow for a true directed network. Using this model we would reapply link prediction algorithms in four different contexts: (1) connecting artists to users for publicity, (2) connecting artists to artists for collaborations, (3) connecting users to artists for finding new music, and (4) connecting users to users as either a social network or a way to find people to see a show with. As far as the rising artist prediction pertains, we have multiple methods and metrics we would like to try, but we would love to extract a stronger community and track changes over time in a temporal version of the model. A temporal network would provid us with the ability to make many other predictions and labelings as well, such as identifying one-hit-wonders, all-time greats, and even our desired next-top-artist. Much of the ground work is in place for this future work, and we look forward to applying what we've learned to move forward and attain our desired results. 

\section*{References}
\small

[1] Boyd, d. m. and Ellison, N. B. (2007), Social Network Sites: Definition, History, and Scholarship. Journal of Computer-Mediated Communication, 13: 210–230. doi:10.1111/j.1083-6101.2007.00393.x

[2] Clauset, A 2017, Lecture 9: Sampling, lecture notes, Network Analysis and Modeling, University of Colorado Boulder, delivered 31 October 2017


\end{document}
